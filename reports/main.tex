%%%%%%%%%%%%%%%%%%%%%%%%%%%%%%%%%%%%%%%%%%%%%%%%%%%%%%%%%%%%%%%%%%%%%%%%%%%%%%%%
%2345678901234567890123456789012345678901234567890123456789012345678901234567890
%        1         2         3         4         5         6         7         8

\documentclass[letterpaper, 10 pt, conference]{ieeeconf}  % Comment this line out
                                                          % if you need a4paper
%\documentclass[a4paper, 10pt, conference]{ieeeconf}      % Use this line for a4
                                                          % paper

\IEEEoverridecommandlockouts                              % This command is only
                                                          % needed if you want to
                                                          % use the \thanks command
\overrideIEEEmargins
% See the \addtolength command later in the file to balance the column lengths
% on the last page of the document

\usepackage[utf8]{inputenc}
\usepackage[T1]{fontenc}

% The following packages can be found on http:\\www.ctan.org
%\usepackage{graphics} % for pdf, bitmapped graphics files
%\usepackage{epsfig} % for postscript graphics files
%\usepackage{mathptmx} % assumes new font selection scheme installed
%\usepackage{mathptmx} % assumes new font selection scheme installed
%\usepackage{amsmath} % assumes amsmath package installed
%\usepackage{amssymb}  % assumes amsmath package installed
\usepackage{graphicx}
\usepackage{booktabs}
\usepackage{multirow}
% \usepackage{CJKutf8} % kept commented: template compatibility, unused in final English text

\title{\LARGE \bf
Billiards Agents with Geometry-Guided Heuristics: Design, Evaluation, and Analysis
}

% Author block: fill real names/affiliations as required by course; keep template structure.
\author{Mingxi Lyu$^{1}$% <-this % stops a space
\thanks{*This work was completed as the final project for AI3603, Fall 2025, Shanghai Jiao Tong University.}% <-this % stops a space
\thanks{$^{1}$School of Electronic Information and Electrical Engineering, Shanghai Jiao Tong University. Email: uke\_sjtu@sjtu.edu.cn}%
}


\begin{document}
\maketitle
\thispagestyle{empty}
\pagestyle{empty}


%%%%%%%%%%%%%%%%%%%%%%%%%%%%%%%%%%%%%%%%%%%%%%%%%%%%%%%%%%%%%%%%%%%%%%%%%%%%%%%%
\begin{abstract}
Billiards (eight-ball) combines continuous control, multi-body collisions, and strict foul rules. This report documents geometry-guided agents (GeometryAgent) and variants designed for the AI3603 course project. We compose defense, cue-ball positioning, and adaptive parameter tuning into V1/V2/V3/Adaptive/Plus series, benchmarked against BasicAgent and BasicAgentPro under a fair four-game rotation across multiple seeds. Geometry heuristics reach $>0.75$ win rate versus BasicAgent and retain advantage versus BasicAgentPro without heavy learning. A PPO attempt was explored but excluded from the final solution due to unstable gains. TODO: insert exact win-rate statistics and confidence intervals once finalized.
\end{abstract}


%%%%%%%%%%%%%%%%%%%%%%%%%%%%%%%%%%%%%%%%%%%%%%%%%%%%%%%%%%%%%%%%%%%%%%%%%%%%%%%%
\section{INTRODUCTION}
Eight-ball billiards requires continuous control (force, angles, offsets), collision reasoning, and strict foul handling. We target strong agents without heavy training and evaluate under a fair four-game rotation (alternating solids/stripes and break order) across multiple seeds. Contributions:
\begin{itemize}
    \item A unified GeometryAgent framework composing defense, cue-ball positioning, and adaptive tuning modules.
    \item Systematic variants (V1/V2/V3/Adaptive/Plus) revealing how defense/position weights, thresholds, and sampling density affect win rate, fouls, and scoring.
    \item A reproducible evaluation setup producing structured logs (\texttt{results.json}, \texttt{metrics.json}) and scripts for visualization.
\end{itemize}

\section{METHOD}
\subsection{Environment and Rules}
Pooltool eight-ball simulator; state contains 3D position/velocity/spin for all balls; action is $(V0, \phi, \theta, a, b)$. Four-game rotation removes first-move and ball-type bias. Foul conditions include white-ball pocketed, first-hit wrong ball, no pocket/no rail, and illegal black-8 shots.

\subsection{GeometryAgent Framework}
Candidates are generated for each target-ball/pocket pair, then densely sampled around ideal geometry (configurable $n\_candidates$, angle\_spread, v0\_spread). Feasibility checks blockages and pocket alignment; physics simulation prunes high-risk shots before selection.

\subsection{Strategy Modules}
\textbf{V2 Defense}: threat estimation and safety shots, controlled by \texttt{defense\_weight} and \texttt{defense\_threshold}.\\
\textbf{V3 Position}: cue-ball leave prediction and run-up scoring weighted by \texttt{position\_weight}.\\
\textbf{Adaptive/Plus}: dynamic adjustment of attack/defense thresholds and v0\_spread to shift aggression based on game state.

\subsection{Baselines and RL Attempt}
Baselines: BasicAgent, BasicAgentPro. PPO (3-layer policy, GAE) was trained but did not surpass Geometry baselines; included as a negative result.

\section{EXPERIMENTAL SETUP}
\begin{itemize}
    \item Evaluation: \texttt{src/train/evaluate.py}, 120 games per matchup, seeds $\{42,123,2024\}$; outputs \texttt{results.json} and \texttt{metrics.json}.
    \item Configs: Geometry V1/V2/V3/Adaptive plus tuned/plus/slim variants; opponents BasicAgent/Pro. Key hyperparameters summarized in Table~\ref{tab:configs} (TODO).
    \item Metrics: win rate, average shots, own/enemy pockets, foul breakdown (white-in, first-hit foul, no-rail, no-hit), remaining balls at termination.
    \item Fairness: four-game rotation; fixed seeds; replay disabled for batch stability.
\end{itemize}

\begin{table}[h]
\caption{TODO: Key hyperparameters for Geometry variants (defense/position weights, thresholds, $n\_candidates$).}
\label{tab:configs}
\begin{center}
\begin{tabular}{|c|c|c|c|}
\hline
Variant & Defense Share & Position Wt. & Attack Thres.\\
\hline
V2\_tuned & TODO & -- & TODO\\
V3\_plus & TODO & TODO & TODO\\
Adaptive\_plus & TODO & TODO & TODO\\
\hline
\end{tabular}
\end{center}
\end{table}

\section{RESULTS}
\subsection{Win Rates}
Geometry variants beat baselines; V3 and tuned/plus reach $\sim$0.75 vs BasicAgent and keep advantage vs BasicAgentPro. Multi-seed mean$\pm$std in Table~\ref{tab:winrate} (TODO). Fig.~\ref{fig:winrate} shows per-run win rates/box plots.

\subsection{Fouls and Scoring Patterns}
V3\_plus lowers foul counts but slightly reduces own pockets; seed variance remains. Adaptive\_plus (reduced v0\_spread) decreases fouls and modestly lifts median win rate. V2\_slim reduces fouls but suppresses opponent less, lowering win rate relative to V2\_tuned. Fig.~\ref{fig:foul} visualizes fouls vs. own pockets.

\subsection{Negative Result: PPO}
PPO training/eval curves (Fig.~\ref{fig:ppo}, TODO) show instability and no consistent gain over Geometry, so it is not part of the final submission.

\begin{figure}[thpb]
      \centering
      \includegraphics[width=0.48\textwidth]{../assets/win_rates.png}
      \caption{TODO: Win rates across runs (grouped by seed or box plot).}
      \label{fig:winrate}
\end{figure}

\begin{figure}[thpb]
      \centering
      \includegraphics[width=0.48\textwidth]{../assets/fouls_vs_own.png}
      \caption{TODO: Fouls vs. own pockets (per-game averages).}
      \label{fig:foul}
\end{figure}

\begin{figure}[thpb]
      \centering
      \includegraphics[width=0.48\textwidth]{../assets/ppo_curves.png}
      \caption{TODO: PPO training/evaluation curves (win rate vs. basic/geometry).}
      \label{fig:ppo}
\end{figure}

\begin{table}[h]
\caption{TODO: Win rate mean $\pm$ std over seeds (BasicAgent/Pro).}
\label{tab:winrate}
\begin{center}
\begin{tabular}{|c|c|c|}
\hline
Variant & vs BasicAgent & vs BasicAgentPro\\
\hline
V3\_position & TODO & TODO\\
V2\_tuned & TODO & TODO\\
Adaptive\_plus & TODO & TODO\\
\hline
\end{tabular}
\end{center}
\end{table}

\section{DISCUSSION}
\subsection{Ablation Insights}
\begin{itemize}
    \item Defense weight: higher values suppress opponent pockets but can raise fouls; sweet spot observed around 0.28--0.32.
    \item Position weight: improves run-outs but over-penalizes risky shots if too high; moderate 0.24--0.26 balances scoring and safety.
    \item Sampling density: increasing $n\_candidates$ and angle\_spread improves quality with diminishing returns beyond 36 candidates.
\end{itemize}

\subsection{Fairness and Reproducibility}
Four-game rotation eliminates first-move/ball-type bias; fixed seeds and 120-game sets reduce variance. Structured logs enable post-hoc checks; figures are generated from \texttt{results.json}/\texttt{metrics.json} via \texttt{scripts/plot_results.py}.

\section{CONCLUSION}
Geometry-guided heuristics deliver strong, stable performance without learning. Defense and cue-ball positioning are primary drivers of gains; adaptive tuning helps when calibrated but harms when overly aggressive. PPO remains a future opportunity under larger budgets or improved reward shaping.

\section{REPRODUCIBILITY NOTES}
Experiments run with \texttt{src/train/evaluate.py}; configs in \texttt{configs/}; outputs in \texttt{experiments/} with seeds $\{42,123,2024\}$. Metrics/plots regenerated via \texttt{scripts/plot_results.py} (requires matplotlib). TODO: list exact commands and dependency versions.

\section*{ACKNOWLEDGMENT}
TODO: Fill in course, instructor, collaborators.

\addtolength{\textheight}{-12cm}   % This command serves to balance the column lengths
                                  % on the last page of the document manually. It shortens
                                  % the textheight of the last page by a suitable amount.
                                  % This command does not take effect until the next page
                                  % so it should come on the page before the last. Make
                                  % sure that you do not shorten the textheight too much.

%%%%%%%%%%%%%%%%%%%%%%%%%%%%%%%%%%%%%%%%%%%%%%%%%%%%%%%%%%%%%%%%%%%%%%%%%%%%%%%%


Appendixes should appear before the acknowledgment.

\section*{ACKNOWLEDGMENT}

The preferred spelling of the word ``acknowledgment'' in America is without an ``e'' after the ``g''. Avoid the stilted expression, ``One of us (R. B. G.) thanks . . .''  Instead, try ``R. B. G. thanks''. Put sponsor acknowledgments in the unnumbered footnote on the first page.



%%%%%%%%%%%%%%%%%%%%%%%%%%%%%%%%%%%%%%%%%%%%%%%%%%%%%%%%%%%%%%%%%%%%%%%%%%%%%%%%

References are important to the reader; therefore, each citation must be complete and correct. If at all possible, references should be commonly available publications.



\begin{thebibliography}{99}

\bibitem{c1} G. O. Young, ``Synthetic structure of industrial plastics (Book style with paper title and editor),'' 	in Plastics, 2nd ed. vol. 3, J. Peters, Ed.  New York: McGraw-Hill, 1964, pp. 15--64.
\bibitem{c2} W.-K. Chen, Linear Networks and Systems (Book style).	Belmont, CA: Wadsworth, 1993, pp. 123--135.
\bibitem{c3} H. Poor, An Introduction to Signal Detection and Estimation.   New York: Springer-Verlag, 1985, ch. 4.
\bibitem{c4} B. Smith, ``An approach to graphs of linear forms (Unpublished work style),'' unpublished.
\bibitem{c5} E. H. Miller, ``A note on reflector arrays (Periodical styleÑAccepted for publication),'' IEEE Trans. Antennas Propagat., to be publised.
\bibitem{c6} J. Wang, ``Fundamentals of erbium-doped fiber amplifiers arrays (Periodical styleÑSubmitted for publication),'' IEEE J. Quantum Electron., submitted for publication.
\bibitem{c7} C. J. Kaufman, Rocky Mountain Research Lab., Boulder, CO, private communication, May 1995.
\bibitem{c8} Y. Yorozu, M. Hirano, K. Oka, and Y. Tagawa, ``Electron spectroscopy studies on magneto-optical media and plastic substrate interfaces(Translation Journals style),'' IEEE Transl. J. Magn.Jpn., vol. 2, Aug. 1987, pp. 740--741 [Dig. 9th Annu. Conf. Magnetics Japan, 1982, p. 301].
\bibitem{c9} M. Young, The Techincal Writers Handbook.  Mill Valley, CA: University Science, 1989.
\bibitem{c10} J. U. Duncombe, ``Infrared navigationÑPart I: An assessment of feasibility (Periodical style),'' IEEE Trans. Electron Devices, vol. ED-11, pp. 34--39, Jan. 1959.
\bibitem{c11} S. Chen, B. Mulgrew, and P. M. Grant, ``A clustering technique for digital communications channel equalization using radial basis function networks,'' IEEE Trans. Neural Networks, vol. 4, pp. 570--578, July 1993.
\bibitem{c12} R. W. Lucky, ``Automatic equalization for digital communication,'' Bell Syst. Tech. J., vol. 44, no. 4, pp. 547--588, Apr. 1965.
\bibitem{c13} S. P. Bingulac, ``On the compatibility of adaptive controllers (Published Conference Proceedings style),'' in Proc. 4th Annu. Allerton Conf. Circuits and Systems Theory, New York, 1994, pp. 8--16.
\bibitem{c14} G. R. Faulhaber, ``Design of service systems with priority reservation,'' in Conf. Rec. 1995 IEEE Int. Conf. Communications, pp. 3--8.
\bibitem{c15} W. D. Doyle, ``Magnetization reversal in films with biaxial anisotropy,'' in 1987 Proc. INTERMAG Conf., pp. 2.2-1--2.2-6.
\bibitem{c16} G. W. Juette and L. E. Zeffanella, ``Radio noise currents n short sections on bundle conductors (Presented Conference Paper style),'' presented at the IEEE Summer power Meeting, Dallas, TX, June 22--27, 1990, Paper 90 SM 690-0 PWRS.
\bibitem{c17} J. G. Kreifeldt, ``An analysis of surface-detected EMG as an amplitude-modulated noise,'' presented at the 1989 Int. Conf. Medicine and Biological Engineering, Chicago, IL.
\bibitem{c18} J. Williams, ``Narrow-band analyzer (Thesis or Dissertation style),'' Ph.D. dissertation, Dept. Elect. Eng., Harvard Univ., Cambridge, MA, 1993. 
\bibitem{c19} N. Kawasaki, ``Parametric study of thermal and chemical nonequilibrium nozzle flow,'' M.S. thesis, Dept. Electron. Eng., Osaka Univ., Osaka, Japan, 1993.
\bibitem{c20} J. P. Wilkinson, ``Nonlinear resonant circuit devices (Patent style),'' U.S. Patent 3 624 12, July 16, 1990. 






\end{thebibliography}




\end{document}
